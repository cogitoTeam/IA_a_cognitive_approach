%--------------------------------------------------------------------------------
% A COGNITIVE SYSTEM
%--------------------------------------------------------------------------------
\subsection{A "cognitive" system}
\begin{frame}{Introduction}{A "cognitive" system}

"How can FCA  optimise a cognitive memory  structure?"

Cognition:
"The mental action or process of acquiring knowledge and understanding through thought, experience, and the senses."
Oxford Dictionary

\end{frame}

%--------------------------------------------------------------------------------
% LEARNING FROM EXPERIENCE
%--------------------------------------------------------------------------------
\subsection{Pattern-matching for knowledge generalisation}
\begin{frame}{Introduction}{Pattern-matching for knowledge generalisation}

An unfamilliar object can be evaluated by looking at attributes shared with objects for which the value is known.

Objects pass their value to their attributes, attributes pass this value on to other objects.

\end{frame}

%--------------------------------------------------------------------------------
% SYTEM LIMITATIONS
%--------------------------------------------------------------------------------
\subsection{System limitations}
\begin{frame}{Introduction}{System limitations}

The value of an object may derive from a specific combination of attributes which, independtly, would not have 
any particular value.

\end{frame}

%--------------------------------------------------------------------------------
% HOW FCA CAN HELP
%--------------------------------------------------------------------------------
\subsection{How FCA can help}
\begin{frame}{Introduction}{How FCA can help}

FCA allows as to evaluate concepts as well as objects and attributes, and so to reason at a higher level of
abstraction.

\end{frame}