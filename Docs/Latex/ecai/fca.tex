%-------------------------------------------------------------------------------
% ADDED INSIGHT
%-------------------------------------------------------------------------------
\subsection{Added insight}
\begin{frame}{What FCA brings to the table}{Added insight}

\Large{What concepts contribute to a winning board? A losing board?}

\end{frame}

%-------------------------------------------------------------------------------
% ELIMINATIONS OF REDUNDANCY
%-------------------------------------------------------------------------------
\subsection{Reducing redundancy}
\begin{frame}{What FCA brings to the table}{Reducing redundancy}

\begin{itemize}
\item Attributes always found together can be merged,
\item<2-> beware of introducing errors!
\end{itemize}

\begin{figure}[ht]
  \begin{minipage}[t]{0.3\linewidth}
    \vspace{0pt}
    \centering
    \includegraphics[width=\textwidth]{img/fca/duck1}
    \\ \color{green}{\footnotesize $hasBill(x) \wedge isDuck(x)$}
  \end{minipage}
  \hfill
  \begin{minipage}[t]{0.3\linewidth}
    \vspace{0pt}
    \centering
    \includegraphics[width=\textwidth]{img/fca/duck2}
    \\ \color{green}{\footnotesize $hasBill(y) \wedge isDuck(y)$}
  \end{minipage}
  \hfill
  \pause
  \begin{minipage}[t]{0.3\linewidth}
    \vspace{0pt}
    \centering
    \includegraphics[width=\textwidth]{img/fca/platypus}
    \\ \color{red}{\footnotesize $hasBill(y) \wedge \neg{}isDuck(y)$}
  \end{minipage}
\end{figure}

\end{frame}

%-------------------------------------------------------------------------------
% FASTER QUERIES
%
%% on peut en dégager une hiérarchie des RPBS. Celle-ci pourrait
%% permettre une recherche plus intelligente des RPBS (réduction du
%% nombre d'homomorphismes à appliquer sur chaque plateau) et donc un
%% gain de temps pour la prise de décision sans devoir réduire
%% drastiquement le nombre de RPBS. On peut imaginer que le système
%% effectuerait une recherche approfondie à posteriori afin de ne pas
%% biaiser l'apprentissage. (L'idée générale est, pendant la phase de
%% jeu, de classer le CBS à évaluer dans un concept et de lui
%% attribuer le point associé à celui-ci. Ce qui permettrait de
%% profiter de la structure du treillis pour diminuer drastiquement le
%% nombre de RPBS rechercher. À savoir que c'est l'étape la plus
%% coûteuse.)
%-------------------------------------------------------------------------------
\subsection{Faster queries}
\begin{frame}{What FCA brings to the table}{Faster queries}

\begin{block}{Structured data}
  \begin{itemize}
    \item Lattice provides a partial-order of configurations,
    \item fewer homomorphismes are required,
    \item intermittent extensive searches should prevent bias.
  \end{itemize}
\end{block}

\end{frame}

%-------------------------------------------------------------------------------
% CHOOSING NEW CONFIGURATIONS 
%
%% Les RPBS introduit dans des concepts
%% parents d'un concept commun (introduisant au moins un CBS), ont
%% potentiellement une sous-partie commune qu'il peut-être intéressant
%% d'extraire comme un nouveau RPBS. (voir common_part.png)
% -------------------------------------------------------------------------------
\subsection{Generating configurations}
\begin{frame}{What FCA brings to the table}{Generating configurations}
\vspace{0.05\paperheight}
Use the hierarchy to choose new relevant partial board-states.
\vspace{0.02\paperheight}
\begin{figure}[ht]
  \includegraphics[height=0.60\paperheight]{img/fca/fca_new_configuration}
\end{figure}

\end{frame}