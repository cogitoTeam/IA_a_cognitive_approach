\documentclass[a4paper,french,10pt]{article}
\usepackage[utf8]{inputenc}
\usepackage[french]{babel}
\addtolength{\oddsidemargin}{-2,5cm}
\addtolength{\textwidth}{5cm}
\addtolength{\topmargin}{-2,5cm}
\addtolength{\textheight}{4cm}

\title{\textbf{Projet TER Master Informatique \\ Analyse et implémentation d'un modèle cognitif} \\ Équipe COGITO}
\author{William Dyce \and Thibaut Marmin \and Namrata Patel \and Clément Sipieter}
\date{Avril 2012}

\begin{document}

\maketitle

\textbf{\LARGE Bilan de pré-soutenance}
\section{Composition de l'équipe COGITO}
L'équipe COGITO est composée de quatre étudiants en provenance des formations DECOL et IMAGINA :
\begin{center}
	\begin{tabular}{l c l}
	William Dyce, &\textit{IMAGINA}, &\texttt{<wilbefast@gmail.com>} \\
	Thibaut Marmin, &\textit{DECOL}, &\texttt{<marminthibaut@gmail.com>} \\
	Namrata Patel, &\textit{DECOL}, &\texttt{<namrata10@gmail.com>} \\
	Clément Sipieter, &\textit{DECOL}, &\texttt{<csipieter@gmail.com>} \\
	\end{tabular}
\end{center}

La composition de l'équipe n'a pas été modifiée depuis le début du projet.

Nous avons été encadrés par :
\begin{center}
	\begin{tabular}{l l}
	Violaine Prince, &\texttt{<prince@lirmm.fr>} \\
	Guillaume Tisserant, &\texttt{<tisserant@gmail.com>} \\
	\end{tabular}
\end{center}
\section{Tâches effectuées}

\begin{tabular}{l l c}
\textbf{Tâche} & \textbf{Effectif} & \textbf{Durée} \\
Analyse du modèle original & Équipe complète & 0,5 mois \\
Formalisation du modèle opérationnel & Équipe complète & 1,5 mois \\
Spécifications du modèle opérationnel & Équipe complète & 0,5 mois \\
Implémentation du modèle opérationnel & & 1 mois \\
\hspace{1cm} \textit{Environnement} & William &  \\
\hspace{1cm} \textit{Module d'analyse} & Narmata & \\
\hspace{1cm} \textit{Mémoire \& persistance} & Thibaut & \\
\hspace{1cm} \textit{Module de raisonnement} & Clément & \\
Tests \& optimisations & Équipe complète & 0,25 mois \\
Documents \& Présentation & Équipe complète & 0,5 mois \\
\end{tabular}

\section{Problèmes rencontrés}
\begin{itemize}
\item Pas de cahier des charges défini dans le sujet du TER, les objectifs du projet ont donc dû être définis, ce qui implique une phase d'analyse et de formalisation conséquente,
\item Difficulté de se réunir à cause des formations diverses des membres de l'équipe,
\item Impossibilité de lister les tâches et de fournir un diagramme de GANTT dans les délais (début Mars),
\item Connaissances nécessaires à la réalisation du projet acquises lors des UEs de ce même semestre.
\end{itemize}
\end{document}
