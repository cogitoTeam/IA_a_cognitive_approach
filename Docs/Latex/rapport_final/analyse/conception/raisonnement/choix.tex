\subsubsection{Cadre Général}

Ce module entre en action après stimulation de la phase d'analyse qui lui fournit, par l'intermédiaire de la mémoire, l'environnement courant et un ensemble de formes reconnues. Le \emph{moteur de choix} se sert alors de la probabilité d'apparition des annotations associées aux formes reconnues afin d'annoter l'environnement courant.

\subsubsection{Application aux jeux de plateau}

Dans le cadre des jeux de plateau, le moteur de choix obtient un ensemble de plateaux, chacun associé à un ensemble de formes reconnues. Chaque plateau correspond au plateau résultant d'un coup possible. On peut donc dire que ce moteur doit choisir entre les différents \og futurs possibles \fg{}.

Afin d'évaluer la probabilité de gain d'un plateau, celui-ci fera la moyenne des probabilités de gain des formes reconnues sur ce plateau. Il choisit finalement l'environnement qui maximise la probabilité de gain.

\begin{figure}[H] 
  \begin{center}
		\includegraphics[width=0.3\textwidth]{files/raisonneur/moteur_de_choix} 
	\end{center}
\caption{Représentation graphique de l'environnement} 
\label{img_env}
\end{figure}
