\subsection{ \og Stigmergie \fg{} }

À fin de bien séparer la logique liée à la réflexion de celui qui structure le jeu, nous suivrons le paradigme \og Agent et Environnement \fg{}. Notre intelligence artificielle sera assimilée à un agent qui évolue dans un environnement qu'il peut observer et modifier. 
Toute communication entre agents sera donc fait par \og Tableau Noir \fg{}, c'est à dire en laissant des traces dans cet environnement partagé. Cette interaction dite \og Stigmergique \fg{} est nécessairement asynchrone mais possède l'avantage de ne nécessiter qu'un seul canal de communication par agent.
Ce canal, reliant l'agent à son environnement, est bidirectionnelle: il permet à l'agent de recevoir des stimuli et d'envoyer des impulsions. Par la suite nous parlerons de "percepts" et de "actions":

%			|      <------ PERCEPTS -------            |
% AGENT     |                                          |       ENVIRONMENT
%			|      ------- ACTION   ------->           |

En séparant l'agent de son environnement nous gagnons beaucoup de flexibilité: l'agent pourrait être aussi bien un humain qu'une machine, machine pouvant être aussi bien notre système qu'un autre. Ceci permettra de tester l'échaffodage de l'application en cours de route sans avoir besoin 


D'autre part l'implémentation de l'agent ne sera pas forcé à suivre celui de l'environnement et vice-versa. Étant donnée que l'Agent n'interagit avec l'environnement qu'à travers ses percepts et ses actions, une interface simple permettrait l'interaction entre agents et environnements, peu importe leur fonctionnement interne précise.

\subsection{ Agent \og Arbitre \fg{} }

