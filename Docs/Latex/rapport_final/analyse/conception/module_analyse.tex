Le rôle du module d'analyse est de traduire la représentation de l'environnement dans un format compréhensible par le système pour ensuite y reconnaître des formes. Ces deux étapes sont gérées respectivement par les deux sous-modules : l'\emph{analyseur conceptuel de base} et l'\emph{analyseur conceptuel poussé}.
\subsection{L'analyseur conceptuel de base}\label{def:analyseur de base}
Lorsque \cogito{} reçoit un plateau en entrée , le \emph{Rule-Book} transforme ce plateau en un paquet qui correspond à l'entrée de l'ACB\footnote{Analyseur Conceptuel de Base.}. Ce paquet contient un plateau courant et l'ensemble des plateaux résultants des coups possibles. Une première analyse d'un plateau au format matriciel lui permet d'extraire des faits représentant la configuration de celui-ci. L'ACB crée alors un paquet correspondant au paquet fournit par l'environnement en convertissant ainsi chaque plateau de ce paquet en un ensemble de faits. Elle passe ensuite ce paquet à l'analyseur conceptuel poussé. 
\subsection{L'analyseur conceptuel poussé}\label{def:analyseur pousse}
L'ACP\footnote{Analyseur Conceptuel Poussé.} a pour but d'associer à chaque plateau du paquet reçu en entrée, une liste de formes pertinentes reconnues, par un mécanisme de reconnaissance de formes. Il procède par une recherche de formes déjà connues (stockées en mémoire) dans chacun des plateaux présents dans ce paquet. Une fois qu'une forme est reconnue, elle est ajoutée à la liste de formes pertinentes associée à ce plateau. L'analyseur enrichit ainsi le paquet fournit par l'analyseur de base.

L'analyseur passe enfin ce paquet enrichi à la mémoire et stimule le module de raisonnement afin que celui-ci commence son processus. 