\subsubsection{Activité purement cognitive}
Nous avions déjà parlé de la simulation des parties bases du modèle de cognition, à savoir l'inconscient. La restriction du domaine d'application au jeu de plateau permet de l'ignorer presque complètement. Certes l'intuition peut jouer un rôle dans le jeu, mais ce défi reste principalement un travail de réflexion d'ordre conscient. Nous n'aurions notamment besoin ni de reflexes, ni de mémoire procédurale.

\subsubsection{Activité cognitive complète}
Gagner un jeu de plateau n'est pourtant pas un travail simple. Il ne suffit pas de prendre le \og meilleur \fg{} coup à chaque tour: pour bien joueur il faut que nous soyons capables de \emph{modéliser} notre adversaire à fin de préduire ses coups, voir de modéliser le modèle qu'il se fait de nous. Il faut ensuite être capable d'utiliser ce modèle pour élaborer une stratégie, donc de \emph{plannifier}. Nous devions \emph{apprendre} suite à erreure si nous ne voulions pas tomber constamment dans les même pièges. 

\subsubsection{Abondance d'études théoriques}
%C'est gr

\subsubsection{Évaluations et comparaison faciles}


\subsubsection{Convergence entre spécialités }
Dans les années précédentes les Masters DECOL\footnote{ Master DECOL : \og Données Connaissances et Langues \fg{} } et IMAGINA\footnote{ Master IMAGINA : \og Images Games and Intelligent Agents \fg{} } étaient rassemblés en un seul appelé I2A\footnote { Maste I2A : \og Ingénierie de l'Intelligence Artificielle \fg{} }.
Nous étions plusieurs dans ce groupe à devoir choisir un camp ou l'autre, donc un projet sur la cognition appliqué aux jeux présentait un très bon moyen de rassembler nos connaissances et de travailler ensemble sur l'intelligence artificielle généralisé.