\section{Charge de travail}
La charge de travail est directement lié à plusieurs critères : le temps, l'effectif disponible et les compétences des membres.

L'équipe COGITO que nous composons et qui a réalisé se projet est composée de quatre étudiants en Master informatique première année (DECOL\footnote{Master DECOL : Données Connaissances et Langage Naturel.} et IMAGINA\footnote{Master IMAGINA : Image, Game and Intelligent Agents.}).

Certaines compétences nécessaires à la réalisation de ce travail n'étaient pas acquise lors du démarrage du projet, ce qui demande un temps d'adaptation supplémentaire.

Le projet s'est déroulé du mois de Février au mois d'Avril, soit environ trois mois de réalisation, période durant laquelle doivent être effectuées les tâches suivantes :

\begin{itemize}
\item étude du sujet, 
\item formalisation, 
\item développement, 
\item évaluation,
\item et préparation du rendu (rédaction du présent rapport et préparation de la soutenance).
\end{itemize}

Nous avons donc dû effectuer un gros travail de simplification du modèle initial, afin de ne pas être trop ambitieux, pour nous permettre d'offrir un travail terminé et fonctionnel lors du rendu.


\section{Nos choix}

Lors des réunions, nous avons fait le choix d'alléger le modèle de certaines caractéristiques détaillées ci-après.

\subsection{Généricité du modèle}
\subsection{Simulation des couches basses}
\subsection{Ignorance de la métamnèse}