\emph{Nous avons choisi d'aborder ce sujet d'une manière différente de l'approche classique. Dans un premier temps, nous n'avons pas cherché à répondre à un besoin fonctionnel mais à une interrogation d'ordre philosophique : 
\begin{center}
\og Est-il possible de synthétiser dans une machine des processus cognitifs se rapprochant de ceux de l'être humain? Si oui comment ? \fg{}
\end{center}
C'est une démarche réellement différente de celle proposée par l'intelligence artificielle dite \og moderne \fg{}\footnote{Cf. Russell S. et Norvig P. (2009 (3rd Ed.)), \og Artificial intelligence : a modern approach \fg{}, Prentice Hall, 0-13-604259-7}, la tendance de nos jours étant de suivre une voie plus pragmatique visant à aboutir à des systèmes dits \og rationnels \fg{}, c'est à dire se comportant de manière optimale vis-à-vis d'une mesure de performance donnée. Ce nouveau paradigme s'est constitué en réaction à une IA vieillissante atteinte par l'ampleur de ces ambitions. À l'origine, les équipes de recherche pensaient pouvoir construire rapidement des systèmes égalant les capacités humaines, mais les résultats ne sont pas arrivés à la hauteur de leurs espérances. Leurs investisseurs déçus ont alors provoqué durant les années 1970-1990 les \og hivers de l'IA \fg{} en retirant la majorité de leurs soutiens financiers. Sans oublier cette leçon d'humilité, nous essayerons de renouer avec une approche biomimétique.}
