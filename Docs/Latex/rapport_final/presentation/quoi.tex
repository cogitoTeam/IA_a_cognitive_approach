Notre objectif lors de ce TER sera donc de construire une intelligence artificielle basée sur une approche biomimétique, c'est à dire que nous tenterons d'approcher un problème d'intelligence artificielle à partir de mécanismes connues du fonctionnement du cerveau. Dans un premier temps, nous devrons étudier et assimiler le modèle de la conscience présenté dans le rapport sur lequel se base notre travail. Dans un deuxième temps, il nous faudra décider d'un domaine sur lequel appliquer notre IA. Ensuite, nous formaliserons le modèle général pour l'appliquer au domaine choisi. Enfin, nous implémenterons notre formalisation. Et Finalement, nous prendrons du recul sur notre projet afin d'effectuer un travail d'évaluation de façon objective.
