Nous avons choisi pour ce TER d'avoir une approche différente de l'approche classique. En effet nous n'avons pas cherché à répondre à un besoin matériel, mais plutôt, à une interrogation philosophique : «Est-il possible de synthétiser dans une machine des processus cognitifs se rapprochant de ceux de l'être humain? Si oui comment ?»
C'est une manière de voir la choses bien différente de celle de l'intelligence Artificielle dite "moderne" \footnote{Voir \og Artificial intelligence : a modern approach \fg{} de Stuart Russel et Peter Norvig}. En effet, la tendance de nos jours est de suivre une voie plus pragmatique visant à aboutir à des systèmes dits "rationnels", c'est à dire se comportant de manière optimale vis-à-vis d'une mesure de performance donnée.
Ce nouveau paradigme s'est constitué en réaction contre une bonne vieille IA atteint par l'ampleur de ces ambitions : au début les chercheurs pensaient pouvoir construire relativement rapidement des systèmes égalant les capacités humaines, mais leurs résultats ne sont pas arrivés à la hauteur de leurs espérances. Leurs investisseurs déçus ont alors provoqués les dites \og{}~Hivers de l'IA~\fg{} aux années 1970-1990 en retirant  la majorité de leur soutient financier.
Sans oublier cette leçon d'humilité, nous essayerons de renouer avec une approche biomimétique.