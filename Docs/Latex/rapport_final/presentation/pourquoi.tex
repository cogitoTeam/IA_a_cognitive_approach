L’Homme a toujours cherché à comprendre et à reproduire les mécanismes naturels qui l'entourent. Un des domaines les plus passionnants reste celui de l'étude du cerveau. Qu'il soit humain ou animal, nous restons fascinés par sa capacité à analyser, à comprendre et à généraliser les problèmes que posent ou \og proposent \fg{} l'environnement.

Afin de se rapprocher du fonctionnement du cerveau, nous traiterons le sujet de notre TER\footnote{TER, Travail d'Étude et de recherche} avec les approches suivantes.


\subsection{Une approche par apprentissage}

Une des principales caractéristiques de l'intelligence, et sûrement une des plus importantes, est la capacité d'apprentissage. C'est pourquoi notre IA devra améliorer ses performances avec le temps en se basant sur ses expériences. Nous pouvons citer Alain Bonnet qui dans son livre «~L'intelligence artificielle, promesses et réalités~», paru en 1984, écrit : «~Les programmes devront apprendre  avec l'expérience et s'auto-améliorer sur de simples jugements de leurs performances que les experts humains fourniront. Dans un premier temps, ils pourront améliorer leurs connaissances et dans un deuxième, leurs mécanismes d'utilisation de ces connaissances, c'est à dire leurs stratégies de plus haut niveau.~».


\subsection{Une approche par reconnaissance de formes}

La reconnaissance de formes dans notre environnement est une autre des principales capacités de notre cerveau. En effet, le cerveau interprète les informations visuels qui lui sont transmises et en extrait des concepts ou formes connues. Par exemple, pour un être humain, les formes tels que \og chaise \fg{}, \og porte \fg{}, \og fenêtre \fg{}, etc, sont automatiquement reconnues par notre cerveau.

