\chapter*{Introduction}
%\addcontentsline{toc}{chapter}{Introduction}
\vspace*{\stretch{1}}
Dans le cadre de notre formation, en première année de Master Informatique à l'Université de Montpellier 2, nous avons réalisé un projet TER (Travail d'Étude et de Recherche) encadré par la professeur \mbox{Violaine} \mbox{Prince} et le doctorant \mbox{Guillaume} \mbox{Tisserant}.

Le but de ce TER fut d'implémenter un modèle d'intelligence artificielle basé sur une approche cognitive. L'agent développé devait être capable d'acquérir de nouveaux concepts sémantiques à partir de son environnent courant et de ses expériences passées.

L'intelligence artificielle moderne se basant principalement sur une vision opérationnelle, il nous semblait intéressant de porter une démarche différente avec un modèle proche de la cognition humaine.

Pour ce faire, nous avons débuté notre étude par l'analyse d'un travail théorique proposant une conceptualisation du fonctionnement du cerveau humain, réalisé en 2010 par \mbox{Guillaume} \mbox{Tisserant}, \mbox{Guillaume} \mbox{Maurin}, \mbox{Ndongo} \mbox{Wade} et \mbox{Anthony} \mbox{Willemot} dans le cadre du cours \og Cognition Individuelle et Collective\fg{} proposé par l'offre d'enseignement du Master Informatique de l'université Montpellier 2.
\vspace*{\stretch{3}}