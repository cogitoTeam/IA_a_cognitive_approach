\chapter*{Introduction}
\addcontentsline{toc}{chapter}{Introduction}
\vspace*{\stretch{1}}
Dans le cadre de notre formation, en première année de Master Informatique à l'Université Montpellier 2, nous avons réalisés un projet TER (Travail d'Étude et de Recherche) encadré par le professeur \mbox{Violaine \textsc{Prince}} et le doctorant \mbox{Guillaume \textsc{Tisserant}}.

Le but de ce TER fut d'implémenter un modèle d'intelligence artificiel basé sur une approche cognitive. Le système développé devait être capable d'apprendre en faisant des liens sémantiques entre concepts tirés de son environnent. 

L'intelligence artificielle moderne s'intéressant principalement à être
opérationnel, il nous semblait intéressant d'expérimenter au contraire un modèle proche de la cognition humaine.

Pour ce faire, nous sommes partis d'un modèle théorique présentant une formalisation du fonctionnement du cerveau humain. Ceci fut réalisé en 2010 par \mbox{Guillaume \textsc{Tisserant}}, \mbox{Guillaume \textsc{Maurin}}, \mbox{Ndongo \textsc{Wade}}, \mbox{Anthony \textsc{Willemot}} dans le cadre du cours \mbox{\og Cognition Individuelle et Collective\fg{}} du second année de Master Informatique.

\vspace*{\stretch{3}}