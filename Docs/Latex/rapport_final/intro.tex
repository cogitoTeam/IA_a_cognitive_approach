\chapter*{Introduction}
\addcontentsline{toc}{chapter}{Introduction}
\vspace*{\stretch{1}}
Dans le cadre de notre formation, en première année de Master Informatique à l'Université de Montpellier 2, nous avons réalisés un projet TER (Travail d'Étude et de Recherche) encadré par le professeur \mbox{Violaine \textsc{Prince}} et le doctorant \mbox{Guillaume \textsc{Tisserant}}.

Le but de ce TER fut d'implémenter un modèle d'intelligence artificiel basé sur une approche cognitive. L'agent développé devait être capable d'acquérir de nouveaux concepts sémantiques à partir de son environnent courant et de ses expériences passées.

L'intelligence artificiel moderne s'intéressant principalement à être opérationnel, il nous semblait intéressant d'expérimenter au contraire un modèle proche de la cognition humaine. 

Pour ce faire, nous sommes partis d'une conception théorique du fonctionnement du cerveau humain réalisé en 2010 par \mbox{Guillaume Tisserant}, \mbox{Guillaume Maurin}, \mbox{Ndongo Wade} et \mbox{Anthony Willemot} dans le cadre du cours \og Cognition Individuelle et Collective\fg{} du second année de Master Informatique.

\vspace*{\stretch{3}}