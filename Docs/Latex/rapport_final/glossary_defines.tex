\storeglosentry{NoSQL}{name=NoSQL, description={catégorie de SGBD (récents pour la plupart) qui se différencie du modèle SQL par une représentation des données non relationnelle. La vision NoSQL abandonne certaines fonctionnalités du modèle relationnel standard au profit d'une plus grande scalabilité. NoSQL ne signifie pas \emph{No SQL} mais \emph{Not only SQL} (se veut un complément à SQL et non un concurrent)}}

\storeglosentry{SGBD}{name=SGBD, description={Système de Gestion de Bases de Données}}

\storeglosentry{GPLv3}{name=GPLv3, description={\emph{GNU General Public License} en version 3 (licence libre copyleft)}}

\storeglosentry{GPLv2}{name=GPLv2, description={\emph{GNU General Public License} en version 2 (licence libre copyleft)}}

\storeglosentry{Apache v2}{name=Apache v2, description={Licence libre en version 2 proposée par la \emph{Apache Software Foundation}}}

\storeglosentry{ACID}{name=ACID, description={Propriété ACID d'une transaction : Atomique, Cohérente Isolée et Durable}}

\storeglosentry{BoardMatrix}{name=\class{BoardMatrix}, description={Classe correspondent à un plateau de jeu sous forme matricielle avec un ensemble d'accesseurs adaptés aux jeux de plateau}}

\storeglosentry{Choices}{name=\class{Percept.Choices}, description={Classe héritier de la classe abstraite Percept, contenant un ensemble de pairs action (Action.Move) et résultat (BoarMatrix)}}

\storeglosentry{Choices_FOL}{name=Choices\_FOL, description={// TODO}}

\storeglosentry{Option_FOL}{name=Option\_FOL, description={// TODO}}

\storeglosentry{CompleteBoardState}{name=CompleteBoardState, description={// TODO}}

\storeglosentry{RelevantPartialBoardState}{name=RelevantPartialBoardState, description={// TODO}}

\storeglosentry{game_logic}{name=\package{game\_logic}, description={Bibliothèque composé des classes BoarMatrix, Rules et Game, permettant de définir et de gérer des jeux de plateau génériques}}

\storeglosentry{game_service}{name=\package{game\_service}, description={Serveur arbitre des parties jouées qui répond par un document XML aux requêtes HTTP envoyés par ses clients}}

\storeglosentry{client-humain}{name={client humain}, description={Client HTML 5 permettant à un humain de visualiser et d'interagir avec le serveur jeu à travers un navigateur web. Utilise la technologie AJAX avec jquery}}

\storeglosentry{AJAX}{name={\texttt{AJAX}}, description={Asynchronous Javascript and XML : technologie permettant la mise à jour en continue d'une page web grâce à des requêtes lancés par un script coté client, avec des réponses en XML}}

\storeglosentry{REST}{name={\texttt{REST}}, description={Representational State Transfer : transfert d'un représentatif de l'état courant du serveur, généralement sous forme HTML ou XML. Un échange RESTful a la particularité d'être sans état, donc sans identification du client}}

\storeglosentry{stigmergie}{name={stigmergie}, description={communication indirecte par le biais d'un environnement partagé, utilisé par exemple par les fourmis}}

\storeglosentry{client-machine}{name={client machine}, description={Client HTML implémenté par la classe agent.Frontier qui sert à communiquer avec le serveur jeu pour que nos agents puissent rester synchronisés avec le jeu}}

\storeglosentry{Cbs}{name=Cbs, description={Complete Board state (cf. \vref{subsection_cbs_rpbs})}}

\storeglosentry{Rpbs}{name=Rpbs, description={Relevant partial board state (cf. \vref{subsection_cbs_rpbs})}}