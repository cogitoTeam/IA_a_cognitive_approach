\clearemptydoublepage
\chapter{Résultats \&  discussion}

\section{Résultats}



\section{Discussion}
\subsection{Un treillis en mémoire ?}
Comme nous l'avons vu dans la partie~\ref{conception_memoire_semantique} (page \pageref{conception_memoire_semantique}), la mémoire sémantique entrepose une matrice de booléens associant des formes remarquables à des plateaux. Ces associations correspondent typiquement à la définition d'un contexte, permettant la construction d'un treillis de concepts.

\subsubsection{Analyse de concepts formels}
L'analyse de concepts formels\footnote{FCA (Formal Concept Analysis)} est l'étude de concepts définis de manière formelle, via un contexte.

\paragraph{Le contexte} Il s'agit d'un triplet $(G,M,I)$ avec $G$ et $M$ des ensembles et $I\subseteq G \times M$ des relations de $G$ dans $M$. Les éléments contenus dans $G$ sont appelés \emph{objets} et ceux de $M$ \emph{attributs}. $I$ est une relation entre une object de $G$ et un attribut de $M$, et qui se dit \og l'objet $g$ possède l'attribut $m$ \fg{}. (Source : Wikpiedia\footnote{L'article \og Analyse de concepts formels \fg{} de Wikipedia en Français à considérablement aidé à la rédaction de cette partie. Contenu sous licence CC-BY-SA.})

En application à \emph{COGITO} nous aurions :

\begin{tabular}{r c l}
$G$ & $ = $ & ensemble des plateaux (objets),\\
$M$ & $ = $ & ensemble des formes remarquables (attributs),\\
$I$ & $ = $ & ensemble des relations plateaux / formes remarquables\\
& & définies dans la matrice actuelle.\\
\end{tabular}

