\chapter*{Abstract}
%\addcontentsline{toc}{chapter}{Abstract}
\vspace*{\stretch{1}}

As part of our post-graduate degree \footnote{first year of a Master Computer Science at the University Montpellier 2.} we worked on what is known as a "Study \& Research"\footnote{in French: "Travail d'Étude et de Rercherche" (TER).} project under \mbox{Violaine} \mbox{Prince} and \mbox{Guillaume} \mbox{Tisserant}, respectively Professor and Doctoral student at the "Computer Science, Robotics and Microelectronics Laboratory of Montpellier"\footnote{in French: "Laboratoire d'Informatique, de Robotique et de Microélectronique de Montpellier" (LIRMM).}. 

The goal of this project was to implement an artificial intelligence based on a human cognitive model. Since modern artificial intelligence tends to focus on performance rather than cognition this unconventional approach seemed like an interesting proposition. 
The developped agent would need to be able to acquire new semantic concepts by observing its current environment and remembering past experiences. 

In order to do so we began by analysing a theoretical study of the human brain, conducted in 2010 by \mbox{Guillaume} \mbox{Tisserant}, \mbox{Guillaume} \mbox{Maurin}, \mbox{Ndongo} \mbox{Wade} and \mbox{Anthony} \mbox{Willemot}. This was written as part of the second-year Master of Computer Science course "Cognition in Individuals and Groups" \footnote{in French "Cognition Individuelle et Collective".} at the University Montpellier 2.
\vspace*{\stretch{3}}