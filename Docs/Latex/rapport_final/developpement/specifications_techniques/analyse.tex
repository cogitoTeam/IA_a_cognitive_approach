Le rôle de l'analyseur est de traduire la représentation de l'environnement dans un format compréhensible par le système pour ensuite y reconnaître des formes connues. Pour ce faire, le module réutilise la bibliothèque \package{\gls{game_logic}} ainsi que le paquetage \package{agent}. Les classes principaux utilisés sont :
\begin{itemize}
  \item \class{\gls{Choices}} (\package{agent}),
  \item \class{\gls{BoardMatrix}} (\package{\gls{game_logic}}),
  \item \class{\gls{Choices_FOL}} (\package{\gls{FOL_objects}}),
  \item \class{\gls{CompleteBoardState}} (\package{\gls{FOL_objects}}),
  \item \class{\gls{RelevantPartialBoardState}} (\package{\gls{FOL_objects}}),
\end{itemize}
Le module est divisé en deux parties l'\emph{analyseur conceptuel de base} et l'\emph{analyseur conceptuel poussé} décrites ci-dessous.
 
\subsection{L'analyseur conceptuel de base}
L'analyseur conceptuel de base (comme défini dans la partie~\ref{def:analyseur de base}, page~\pageref{def:analyseur de base}) convertit, plus précisément, une instance de \class{\gls{Choices}} en une instance de \package{\gls{Choices_FOL}} en se servant de classes qui modélisent la logique du premier ordre. 

Une première analyse de plateau au format matriciel (\class{\gls{BoardMatrix}}) permet de générer l'ensemble des faits logiques représentant la configuration de celui-ci. Cet ensemble de faits constitue une base de faits qui est ensuite stockée comme attribut de la classe \texttt{CompleteBoardState} correspondante à ce plateau. L'analyseur crée alors le paquet \texttt{Choices\_FOL} correspondant au paquet \texttt{Choices} fournit par l'environnement en convertissant chaque plateau (\texttt{BoardMatrix}) de \texttt{Choices} en un plateau (\texttt{CompleteBoardState}) de \texttt{Choices\_FOL}. Elle passe ensuite ce paquet à l'analyseur conceptuel poussé. 
\subsection{L'analyseur conceptuel poussé}
Lorsque l'analyseur conceptuel poussé, comme défini dans la partie~\ref{def:analyseur pousse}(page~\pageref{def:analyseur pousse}), reçoit ce paquet, il associe à chaque plateau du paquet, les formes pertinantes reconnues (des \texttt{RelevantPartialBoardState}).

Les formes (\texttt{RelevantPartialBoardState}) récupérées de la mémoire correspondent à des règles logiques représentées comme une conjonction d'atomes, le dernier atome étant la conclusion de cette règle. Ici, l'hypothèse décrit une configuration (forme) pertinante, et la conclusion introduit un identifiant. Par exemple, le fait d'avoir pris un coin est représenté par la règle :

\textit{$isCorner() \wedge is(Mine) \Longrightarrow \_rpbs034$}. 

Ensuite, l'analyseur, en tant que moteur d'inférence, sature la base de faits de chaque plateau rencontré dans le paquet \texttt{Choices\_FOL} par l'application de l'ensemble de ces règles (\texttt{RelevantPartialBoardState}). La reconnaissance des formes dans chacun de ces plateaux revient alors à rechercher l'homomorphisme de l'atome représentant l'identifiant d'une forme pertinente dans la base de faits saturée de ce plateau. Une fois qu'une forme est reconnue (existence d'un homomorphisme), la règle \texttt{ReleventPartialBoardState} est ajoutée à la liste de formes pertinentes associée à ce plateau.

Pour résumer, le rôle de l'analyseur conceptuel poussé est donc de déterminer et d'ajouter cette liste de formes pertinentes à chaque plateau présent dans le paquet \texttt{Choices\_FOL}. 

L'analyseur passe enfin ce paquet enrichi à la mémoire et stimule le module de raisonnement afin que celui-ci commence son processus. 