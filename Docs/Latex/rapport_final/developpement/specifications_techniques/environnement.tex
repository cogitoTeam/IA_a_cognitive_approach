  
  
  
\subsection{Serveur textsc{game_service}}

Le Game Service est un logicielle à part entière.

\subsubsection{"Hyper-text Transport Protocol"}

Le HTTP, protocole pilier du web, a l'avantage surtout d'être très répandu. Son ubiquité assure l'existence de bibliothèques Libres simples et complètes pour l'utiliser, peu importe le langage ou plateforme considéré. 
Les communications Agent-Environnement sont asynchrones 

\subsubsection{"Servlets" Java}

Nous avons choisies la technologie Java Servlet pour implémenter le gestionnaire de jeux. Un Servlet est une classe Java instancié par un serveur telle Tomcat ou Glassfish à fin de répondre à une requête HTTP spécifique. Cette objet est détruit immédiatement après avoir envoyé sa réponse, généralement sous forme de string HTML ou XML.
Solution peu puissance et surtout peu intuitive pour les concepteurs web non-programmeurs, elle est le plus souvent utilisé pour prototyper des sites web dynamiques avant de passer à une technologie plus complet comme JSP ou PHP. Cependant pour notre application il suffit très largement.

\subsection{Client Humain (HTML 5)}
    
\subsection{Client Agent}
    