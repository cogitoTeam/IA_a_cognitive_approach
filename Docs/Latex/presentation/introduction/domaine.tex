%----------------------------------------
% DOMAINE D'APPLICATION : POUQUOI
%----------------------------------------
\begin{frame}{Domaine d'application}{Jeu de plateau}

\begin{block}{Pourquoi le jeu de plateau ?}
\begin{itemize}
\item Convergence \texttt{DECOL} / \texttt{IMAGINA}.
\pause
\item Activité purement cognitif.
\pause
\item Activité cognitif complète.
\pause
\item Évaluation facile de la performance.
\end{itemize}
\end{block}

\end{frame}


%----------------------------------------
% LE MINIMAX
%----------------------------------------
\begin{frame}{Domaine d'application}{Théorie des jeux}

\begin{block}{Théorème du MiniMax}
\begin{itemize}
\item J. Von Neumann, 1928.
\item \textit{Stratégie optimale pour un joueur donné.}
\end{itemize}
\end{block}
\end{frame}


%----------------------------------------
% LIMITES DU MINIMAX
%----------------------------------------

\begin{frame}{Domaine d'application}{Limites du MiniMax}

\begin{block}{Type de confrontation}
\begin{itemize}
\item \underline{Minimax}: Jeux compétitifs, à deux joueurs, à somme nulle.
\item \underline{Minimax \& Nash}: Durée et nombre d'options finis.
\end{itemize}
\end{block}

\pause

\begin{block}{Temps de calcul}
\begin{itemize}
\item En moyenne \emph{$O(b^{\frac{d}{2}})$} :
	\begin{itemize}
	\item d : longeur partie.
	\item b : options par tour.
	\end{itemize}
\item En pratique : besoin d'heuristiques $\Rightarrow$ perte d'optimalité.
\end{itemize}
\end{block}

\end{frame}