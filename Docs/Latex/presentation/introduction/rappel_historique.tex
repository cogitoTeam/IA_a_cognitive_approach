%----------------------------------------
% RAPPEL HISTORIQUE - DEEP BLUE
%----------------------------------------

\begin{frame}{Rappel historique}{Deep Blue}

\begin{block}{Deep Blue} 
\begin{itemize}
\item Programme d'échecs développé par IBM.
\item Victoire contre Garry Kasparov en 1997.
\item Première défaite d'un grand maître sous contraintes normales de temps.
\end{itemize}
\end{block}

\pause

\begin{block}{Robert Levinson} 
\begin{itemize}
\item \textit{\og But doesn't know that it's playing chess.\fg{}}
\item Est-ce donc vraiment de l'intelligence?
\end{itemize}
\end{block}

\end{frame}



%----------------------------------------
% RAPPEL HISTORIQUE - THÉORIE DES JEUX
%----------------------------------------

\begin{frame}{Rappel historique}{Théorie des Jeux}
 
\begin{block}{Théorème du Minimax} 
\begin{itemize}
\item Élaboré par John Von Neumann en 1928.
\item Stratégie optimale pour jeux compétitives tels les échecs.
\end{itemize}
\end{block}

\pause

\begin{block}{Équilibre de Nash} 
\begin{itemize}
\item Définit par John Forbes Nash en 1950.
\item Fondation de la Théorie des Jeux.
\end{itemize}
\end{block}

\pause

\begin{block}{Élagage Alpha-beta} 
\begin{itemize}
\item Conçu par John McCarthy en 1958.
\item Amélioration du Minimax.
\end{itemize}
\end{block}

\end{frame}


%----------------------------------------
% RAPPEL HISTORIQUE - LIMITES
%----------------------------------------

\begin{frame}{Rappel historique}{Limites}

\begin{block}{Domaine d'application}
\begin{itemize}
\item Jeux compétitifs à somme nulle.
\item Durée et nombre d'options finis.
\end{itemize}
\end{block}

\pause

\begin{block}{Temps de calcul}
\begin{itemize}
\item Complexité moyenne $O(b^{\frac{d}{2}})$ avec élagage.
	\begin{itemize}
	\item d : profondeur de l'arbre de décision.
	\item b : facteur de branchement.
	\end{itemize}
\item Utilisation d'heuristiques, donc perte d'optimalité.
\end{itemize}
\end{block}

\end{frame}