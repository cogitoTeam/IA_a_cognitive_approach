%----------------------------------------
% RAPPEL HISTORIQUE - DEEP BLUE
%----------------------------------------

\begin{frame}{Rappel historique}{Deep Blue}

\begin{block}{Deep Blue} 
\begin{itemize}
\item Victoire contre Garry Kasparov en 1997.
\item $1^{\grave{e}re}$ défaite de grand maître sous contraintes normales de temps.
\end{itemize}
\end{block}

\pause

\begin{block}{R. Levinson} 
\begin{itemize}
\item \textit{\og But doesn't know that it's playing chess.\fg{}}
\item Intelligence? Artificielle? Synthétique?
\end{itemize}
\end{block}

\end{frame}



%----------------------------------------
% RAPPEL HISTORIQUE - THÉORIE DES JEUX
%----------------------------------------

\begin{frame}{Rappel historique}{Théorie des Jeux}

\begin{block}{Théorème du Minimax} 
\begin{itemize}
\item J. Von Neumann, 1928.
\item \textit{Stratégie optimale pour un joueur donné.}
\end{itemize}
\end{block}

\pause

\begin{block}{Équilibre de Nash} 
\begin{itemize}
\item J. F. Nash, 1950.
\item \textit{Union de stratégies, localement optimale pour tous.}
\end{itemize}
\end{block}

\end{frame}


%----------------------------------------
% RAPPEL HISTORIQUE - LIMITES
%----------------------------------------

\begin{frame}{Rappel historique}{Limites}

\begin{block}{Domaine d'application}
\begin{itemize}
\item \underline{Minimax}: Jeux compétitifs, à deux joueurs, à somme nulle.
\item \underline{Minimax \& Nash}: Durée et nombre d'options finis.
\end{itemize}
\end{block}

\pause

\begin{block}{Temps de calcul}
\begin{itemize}
\item Complexité moyenne \emph{$O(b^{\frac{d}{2}})$} si trié et élagé:
	\begin{itemize}
	\item d : longeur de la partie.
	\item b : nombre d'options par tour.
	\end{itemize}
\item En pratique: besoin d'heuristiques, donc perte d'optimalité.
\end{itemize}
\end{block}

\end{frame}