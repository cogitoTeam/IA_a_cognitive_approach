\begin{frame}{Analyseur conceptuel}{Généralités}
\begin{itemize}
  \item Représente les connaissances tirées de l'environnement
  \item Analyse ces connaissances afin d'en tirer des nouvelles
\end{itemize}
\end{frame}

\begin{frame}{Analyseur conceptuel}{Analyse détaillée}
\begin{itemize}
  \item Représentation des connaissances : \textbf{vocabulaire}
  \begin{itemize}
    \item Graphes conceptuels de base ou
    \item Formules de logique du premier ordre
  \end{itemize}
  \item Analyse des connaissances : \textbf{méchanisme}
  \begin{itemize}
    \item Interrogation avec la mémoire
    \item Recherche d'homomorphismes
  \end{itemize}
\end{itemize}
\end{frame}

\begin{frame}{Analyseur conceptuel}{Implémentation : Rôles du module}
\begin{itemize}
  \item Convertisseur :
  \begin{itemize}
    \item Rend les données de l'environnement \enquote{lisibles} par l'IA
  \end{itemize}
  \item Moteur d'inférence :
  \begin{itemize}
    \item Applique les règles générés par l'IA afin d'en
    sortir des nouveaux concepts
  \end{itemize}
\end{itemize}
\end{frame}

\begin{frame}{Analyseur conceptuel}{Implémentation : Classes principales}
\begin{itemize}
  \item \textbf {Choices (environnement)} : représente un plateau
  courant, un coup et l'ensemble des plateaux résultants de ce coup
  \item \textbf {BoardMatrix(environnement)} : représente un plateau en forme
  d'une matrice
  \item \textbf {Choices\_FOL (IA)} : version logique du premier ordre de
  Choices (même structure, attributs décrits par des formules logiques)
  \item \textbf {CompleteBoardState (IA)} : version logique du premier ordre de
  BoardMatrix (classe qui décrit la configuration
  d'un plateau complet comme une liste de faits logiques)
  \item \textbf {RelevantPartialBoardState (IA)} : classe qui décrit la
  configuration d'une sous-partie pertinante d'un plateau comme une règle
  logique
\end{itemize}
\end{frame}


\begin{frame}{Analyseur conceptuel}{Implémentation détaillée}
\begin{itemize}
  \item \textbf {Convertisseur} :
  \begin{itemize}
    \item Entrée (de l'environnement): instance de \enquote{Choices}
    \item Algorithme qui transforme un \enquote{BoardMatrix} en un
    \enquote{CompleteBoardState}
    \item Sortie: instance de
    \enquote{Choices\_FOL}
  \end{itemize}
  \item \textbf {Moteur d'inférence} :
  \begin{itemize}
    \item Entrée (de la mémoire) : instance de
    \enquote{RelevantPartialBoardState}
    \item Algorithme de saturation de la base de faits des
    \enquote{CompleteBoardState} par la règle d'entrée
    \item Ajout d'une liste de \enquote{RelevantPartialBoardState} présents dans
    chaque \enquote{CompleteBoardState} du pacquet \enquote{Choices\_FOL}
    \item Sortie (passée à la mémoire) : instance de
    \enquote{Choices\_FOL}
  \end{itemize}
\end{itemize}
\end{frame}
