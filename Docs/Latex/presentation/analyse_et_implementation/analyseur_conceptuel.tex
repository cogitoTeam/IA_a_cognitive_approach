\begin{frame}{Analyseur conceptuel}{Généralités}
\begin{itemize}
  \item Représente les connaissances tirées de l'environnement
  \item Analyse des connaissances afin d'en tirer de nouvelles
\end{itemize}
\end{frame}

\begin{frame}{Analyseur conceptuel}{Analyse détaillée}
\begin{itemize}
  \item Représentation des connaissances : \textbf{vocabulaire}
  \begin{itemize}
    \item Formules logiques (logique du premier ordre)
  \end{itemize}
  \pause
  \item Analyse des connaissances : \textbf{mécanisme}
  \begin{itemize}
    \item Recherche d'homomorphismes
  \end{itemize}
\end{itemize}
\end{frame}

\begin{frame}{Analyseur conceptuel}{Implémentation : rôles du module}
\begin{itemize}
  \item \textbf{Convertisseur :}
  \begin{itemize}
    \item Rend les données de l'environnement \enquote{lisibles} par l'IA
  \end{itemize}
  \item \textbf{Moteur d'inférence :}
  \begin{itemize}
    \item Applique les règles générées par l'IA afin de
    tirer de nouvelles informations
  \end{itemize}
\end{itemize}
\end{frame}

\begin{frame}{Analyseur conceptuel}{Implémentation : Classes principales}
\begin{itemize}
  \item \textbf {Choices (environnement)} : représente un plateau
  courant et l'ensemble des plateaux résultants des coups possibles
  \pause
  \item \textbf {BoardMatrix(environnement)} : représente un plateau en forme
  d'une matrice
  \pause
  \item \textbf {Choices\_FOL (IA)} : version logique du premier ordre de
  Choices (même structure, attributs décrits par des formules logiques)
  \pause
  \item \textbf {CompleteBoardState (IA)} : version logique du premier ordre de
  BoardMatrix (classe qui décrit la configuration
  d'un plateau complet comme une liste de faits logiques)
  \pause
  \item \textbf {RelevantPartialBoardState (IA)} : classe qui décrit la
  configuration d'une sous-partie pertinante d'un plateau comme une règle
  logique 
\end{itemize}
\end{frame}


\begin{frame}{Analyseur conceptuel}{Implémentation détaillée}
\begin{itemize}
  \item \textbf {Convertisseur} :
  \begin{itemize}
    \item Entrée (de l'environnement): instance de \texttt{Choices}
    \pause
    \item Génération de faits logiques représentant la configuration d'un
    plateau en forme matricielle (\texttt{BoardMatrix})
    \pause
    \item Base de faits correspondante stockée dans une instance de
    \texttt{CompleteBoardState}
    \pause
    \item Sortie: instance de
    \texttt{Choices\_FOL}
  \end{itemize}
\end{itemize}
\end{frame}

\begin{frame}{Analyseur conceptuel}{Implémentation détaillée}
\begin{itemize}
\item \textbf {Moteur d'inférence} :
\begin{itemize}
  \item Entrée (de la mémoire) : liste de \texttt{RelevantPartialBoardState},
  règles du type : 
  \pause
  \emph{$isCorner(x) \wedge isMine(x) \Longrightarrow \_rpbs034(x)$}
  \pause
  \item Saturation de la base de faits des \texttt{CompleteBoardState} en
  appliquant l'ensemble de ces règles
  \pause
  \item Reconnaissance de formes : recherche d'homomorphisme entre l'atome de
  conclusion de la règle (\texttt{RelevantPartialBoardState}) et la base de
  faits de chaque plateau
  \pause
  \item Ajout de la liste de \texttt{RelevantPartialBoardState} présents dans
  chaque \texttt{CompleteBoardState} du paquet \texttt{Choices\_FOL}
  \pause
  \item Appel à la méthode \texttt{stimulate} du module de raisonnement
  \item Sortie (passée à la mémoire) : instance de \texttt{Choices\_FOL}
\end{itemize}
\end{itemize}
\end{frame}
