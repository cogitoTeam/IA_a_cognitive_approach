\begin{frame}{Perspectives}

\begin{block}{Rationalité versus conscience}
\begin{itemize}
\item Plus facile d'évaluer la rationalité.
\item Glissement facile vers l'opérationnelle.
\end{itemize}
\end{block}

\pause

\begin{block}{Problèmes rencontrés}
\begin{itemize}
\item Compétences hétérogènes.
\item Big data et NP-Complétude:
\begin{itemize}
\item Recherche d'homomorphismes.
\item Recherche de l'ensemble des sous-graphes.
\end{itemize}
\end{itemize}
\end{block}

\pause

\begin{block}{À suivre}
Évaluation du système à faire ...
\end{block}

\end{frame}

\begin{frame}{Perspectives}{Un treillis de concepts en mémoire ?}
\begin{block}{Le contexte}
Un contexte est un triplet $(G,M,I)$ avec
\begin{itemize}
\item $G$ l'ensemble des objets
\item $M$ l'ensemble des attributs
\item $I \subseteq G \times M$ l'ensemble des relations
\end{itemize}
On pourrait donc utiliser la mémoire sémantique comme contexte pour la création d'un treillis de concepts.
\end{block}
\end{frame}

\begin{frame}{Perspectives}{Un treillis de concepts en mémoire ?}
\begin{block}{Utile ?}
\begin{itemize}
\item Abstraction supplémentaire
\item Travail sur des ensembles de formes (RPBS)
\end{itemize}
\end{block}
\end{frame}

\begin{frame}{Perspectives}{Un treillis de concepts en mémoire ?}
\begin{block}{Exemple (concret)}
\begin{center}
	\only<1|handout:0>{\includegraphics[width=0.2\textwidth]{img/conclusion/espresso_tmp}}
	\only<2->{\includegraphics[width=0.2\textwidth]{img/conclusion/espresso}}
	\only<-2|handout:0>{\includegraphics[width=0.2\textwidth]{img/conclusion/ordi_tmp}}
	\only<3->{\includegraphics[width=0.2\textwidth]{img/conclusion/ordi}}
\end{center}
\begin{center}
	\only<4|handout:0>{\includegraphics[width=0.3\textwidth]{img/conclusion/cafe_renverse_tmp}}
	\only<5->{\includegraphics[width=0.3\textwidth]{img/conclusion/cafe_renverse}}
\end{center}
\end{block}
\end{frame}